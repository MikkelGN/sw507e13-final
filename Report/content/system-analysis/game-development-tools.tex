\section{Game Development Tools}\label{section:game-development-tools}
One of the hard requirements, seen in \secref{section:requirements}, is the development of a game. 
In order to ease the development of this game, two game development tools will be analysed.
The target platform of this project is Windows Phone 8, therefore the tools have to work on this specific platform.
There are many different tools for developing games on the Windows Phone 8 platform, however, in this report the focus is on XNA and Unity.
These tools have been selected due to their popularity as well as documentation.


\subsection{Unity}\label{subsection:unity}
Unity is a software game engine in which it is possible to create a game, where the engine provides functionalities.
Unity has a lot of available features for creating a game, such as game mechanics, 3D modelling environment, and a lot of existing scripts to ease the game development.
While a lot of the mechanics is predefined it still remains possible to edit and create mechanics such as physics within a game.

There are three available programming languages in Unity, namely C\#, JavaScript, and Boo.
Unity uses .NET 2.0 according to \citet{misc:UnityNet}. 
Because Unity uses \textit{monoDeveloper 2.8.2} as its main IDE, Unity uses C\# version 4.0 \citep{misc:MonoVersion}.
It means that some features that are dependable on a .NET framework higher than 2.0 will not work in Unity.
When programming an application in Unity it can be ported to a wide range of platforms, which include Android, iOS, and Windows Phone.


\subsection{XNA}
XNA is a software game development framework in which it is possible to develop games.
%However, unlike Unity, XNA does not support drag and drop or have any visual feedback when creating the environment.
%Thus, when creating an application in XNA, only coding is available for the programmer.
The XNA framework supports the developer with libraries that allow the developer to construct a game, where the game mechanics are written by the developer.
Due to XNA being a framework, it does not have a user interface.

When programming in XNA the programmers are able to program in C\#, Visual Basic, F\#, JavaScript, and C++.
XNA is a Microsoft product and can be ported to any Microsoft platform. 
XNA has some built-in features that makes it easy to make games, such as network communication.
XNA runs on the .NET framework 4.0 \citep{misc:XNAVersion}, which means that XNA can use features supported by the .NET 4.0 framework.

\subsection{Summary}
Both applications provides a variety of features for creating games.
When it comes to game development, both of them are equally fit, as obtaining information from the sensors requires a single instruction \citep{misc:UnitySensor, misc:WindowsPhoneSensor}.
Unity offers a user friendly 3D environment whereas XNA is purely textbased and therefore less user friendly.
Since both of the applications offer C\# as a programming language, this means potentially both could be used for the project, corresponding to the chosen development language specified in \secref{section:limitations}.
Both applications supports various target platforms, but Unity has a wider variety of platforms the application can be ported to.
However, the variety of supported platforms is not a factor for this project as the project is for Windows Phone 8.
In conclusion of this it has been decided to use XNA because it uses a newer version of .NET framework, and since an extensive visual platform is not needed to create a simple game.