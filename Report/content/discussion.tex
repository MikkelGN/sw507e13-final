\chapter{Discussion}\label{chapter:discussion}
While developing the application for this project, multiple problems were found, resulting in inaccurate dead reckoning positioning.
Although many of the problems were solved, some problems still exist, which causes the dead reckoning to be inaccurate.
Possible solutions to the existing problems in the application will be examined and discussed in further development, \secref{section:further-development}.

Apart from the existing problems, implementation decisions, made during development of this application, are discussed in this chapter.
Furthermore, the application will be compared to existing systems to determine the pros and cons of this application.

\subsection*{Platform}
The application that is developed for this project is currently working on Windows Phone 8 platform.
The choice of developing for Windows Phone 8 gives the opportunity to distribute to a large market, where Windows Phone 8 had $4.2$ million users in 2012 \citep{misc:wp8-users}.
The market could, however, be expanded if the application was to be developed for other platforms as well, but the Windows Phone 8 platform still allows for a large market.
However, with the amount of time at hand the resources were allocated to develop a dead reckoning application for a single platform, as the focus was to make dead reckoning work, rather than distributing the application for multiple platforms. 

Furthermore, the variance for the acceleration is determined for one accelerometer and thus the variance is not directly portable to other phones, which may have other variances for the acceleration even if they have the same accelerometer model.
For that reason, it could be argued that when the application is first used, the phone should be placed on a flat surface for a given time period to determine the variance of the acceleration for that specific accelerometer.

\subsection*{One Dimension}
For the scope of this project, dead reckoning has been implemented for one dimension.
It was done to simplify how movements could look, thus making dead reckoning easier to implement.
%It is undesirable to track movement in only one dimension, since it limits the part of the movement traced.

A constraint was that the user should do sideways movements.
However, this constraint was not a problem since the application only needed one dimensional movements and there are a lot of arcade games that just need one dimensional movements.

The movement tracking could be in two dimensions for a planar surface, or even in three dimensions when there is a desire to track vertical movement as well.
Movement tracking in two dimensions, could be possible by keeping track of the phone's heading.

Three dimensions would make tracking more complex, since it would require something to track the movement in three axis.
However, with more time, tracking jumps could potentially be pursued as it would not require additional environmental constraints, other than a place where it is possible to jump freely.

\subsection*{Pedestrian Dead Reckoning}
The application developed for this project was an algorithm using pedestrian dead reckoning to track the user's movements. 
It gave certain opportunities, while removing others.
The fact that pedestrian dead reckoning uses the sensors of the phone itself, with no dependencies from other sources, could be used to track the user's movement.

Pedestrian dead reckoning gave the advantage that extra equipment, such as a Wi-Fi beacon, was not needed, and was the primary reasoning for using pedestrian dead reckoning.
However, it also limited the scope of what information could be used to determine the movement of the user.
A problem when not using such information, was that after some time, the pedestrian dead reckoning became highly inaccurate and the variance would be higher. 
This would result in a higher uncertainty.
However, implementing position tracking by another method could be used as evidence on the Bayesian network. 
This other method could be implemented in conjunction with the pedestrian dead reckoning as an extended Kalman filter to increase the accuracy of dead reckoning. 

\subsection*{Motivation}
Various features were implemented in the game developed, to motivate people to play the game. 
These features included a high score, power-ups, and increasing difficulty.

The high score simply shows the highest scores achieved on the device.
%For the highscore it is possible to see the highest score of the user, and thereby see the progression.
Furthermore, the high score is intended to make the user compete against other users, thereby making people want to play more to improved their score or to be better than their friends.

To make the game entertaining, power ups and increasing ball speed were implemented, since it was believed that without those features the game would become tedious and thus less entertaining. 

As mentioned above, features have been implemented to make the game entertaining and at the same time make people exercise.
Whether the features make the game entertaining and motivates the users towards exercise, has not been examined.
In order to examine this, the target audience of the application would have to be involved to test the game and give feedback on what they find entertaining.
Furthermore, a fitness consultant could be contacted to determine whether the movement patterns needed to play the game is good exercise or not.

\section{Perspective}
With the dead reckoning exercise application developed for this project, the application can be compared to other existing exercise systems that is described in \secref{section:existing-systems}.

The game that has been developed is for exercise purposes.
It is believed that Endomondo and Zombies, Run! have the potential to evoke exercise on a more intensive level, due to the nature of running versus sidestepping.
However, Moves does not in itself lead to exercise as the developed application, but tracks the progress of a whole day and not only when the user wants to exercise.

When using applications that induce or track exercise, the user should gain some value from it.
For this project the primary focus was entertaining exercise, while systems such as Endomondo and Moves gives the value of tracking the exercise performed.
A system that has been examined in this project and gives entertainment value is Zombies, Run!.
Zombies, Run! brings an interesting storyline, written by professional authors and performed by professional narrators, while running.
Furthermore, it entertains the user while exercising, however, if a user does not like the story, the primary entertainment value of the Zombies, Run! is counter productive, as the user would then stop playing due to him perceiving that it is a bad story.
The developed application is for the user to entertain himself by playing the game and achieving a better high score. 
However, therein lies the problem that there is no storyline to make the user motivated to play the game, and thus the game may become a tedious experience.

When doing exercise, information about progress, such as, calories burned or the amount of exercise performed can be presented.
The application was developed to make the user have fun when doing exercise, and thereby motivate people, after having used the application, to begin doing other more intensive types of exercise afterwards.
Therefore, information about calories burned and the amount of exercise performed was not an important part, as it is for Endomondo, Moves, and Zombies, Run!.
Implementing tracking of the amount of exercise performed when using the application could be an additional feature that motivates people.
However, due to the primary focus being dead reckoning and embedded systems, exercise statistics have  not been a concern.

As discussed, the application developed differs in various ways from the existing systems described in \secref{section:existing-systems}, and has had a different approach to exercise.
The developed application is deemed to be different than the other systems.
It serves a different purpose where the user plays with his movement and exercise comes as a bonus, in contrast to the other systems, where exercise has a more central and visible role. 
In that regard, the developed application is not unique when it comes to the game itself, but it is the movement tracking by dead reckoning, working as the controller, which makes the developed application interesting.

\section{Further Development}\label{section:further-development}
During this project, a number of problems have been encountered. 
These problems were discussed in \chapref{chapter:discussion}, ideas and features to solve these problems will be presented in this section.

\subsection{Cross platform}
As discussed earlier in this chapter, this project was focused on Windows Phone 8 using a Nokia Lumia 820. 
However, making the application available on multiple platforms would open up to a bigger market. 
To do this effectively, code could be ported to Unity which was shortly described in \secref{subsection:unity}. 
Unity works with C\# which would make it easy to port the code, as it is able to compile the code for Windows Phone, iPhone, and Android.
Next, a solution to the problem of noise on individual phones must be found. 
This problem could be solved by calibrating when the application is used for the first time.
The calibration would give the application all the information it needs to run correctly on the individual phone. 

\subsection{Rotation Matrix}
%Hvad er det?
%Hvad kan det improve?
%Hvad dårligt giver det programmet?
%Hvordan kan det implementeres?
%Hvorfor kom det ikke med?
The application corrects the tilting of the phone when a yaw-rotation or pitch-rotation is done by the user.
However, when the phone is rotated in both pitch and yaw, the correction of the acceleration is inaccurate.
This inaccuracy happens because the acceleration has been used to determine the angle of the pitch and yaw before the acceleration in itself is corrected.

To correct the acceleration when the phone is tilted in pitch, yaw, and roll, a rotation matrix can be used to multiply on the gravitational pull vector, $\vec{G} = \begin{bmatrix}
0 & 0 & g
\end{bmatrix}$ \cite{misc:rotation-matrix}.

The implementation of the rotation matrix to correct for tilting, has not been done due to time constraints.
However, if it was to be implemented, it would replace the SF node in the dynamic Bayesian network, the atan2 node, and the Comp node would have to find the missing angles. 

In addition, if the rotation matrix was implemented, it could adjust the gravitational pull in three dimension and thereby simplify the change to additional dimensions.

\subsection{Additional Dimensions}
This project investigated dead reckoning in one dimension, but it is possible to expand the application to two or three dimensions. 
Getting dead reckoning to work in two and three dimensions would enable a wide array of new gameplay and, hereby, the variation of exercise. 
Considering two dimensions, it is possible to move forwards, backwards, sideways, and diagonal.  
Since the precision of dead reckoning in one dimension is inaccurate, it is assumed that it is also inaccurate in two dimensions.
To solve the inaccuracy of dead reckoning, observations such as Wi-Fi fingerprinting could be implemented.
Although, when moving in three dimensions, additional problems can arise.

\subsection{Motivational Factors}\label{section:motivational-factors}
To implement a feature where players could compete against their friends, social networks such as \textit{Facebook} could be integrated. 
Social networks would offer a platform from which social relations have already been established and thus simplify the process of the development of the application.
Being able to turn the social network feature on or off would prevent the issue with people being discouraged from playing the game in case they want their game progress to be private. 
Other factors could be developing additional games, such that people had a variety of games to play.
Furthermore, motivational features can always be strived for to make the user exercise more, as discussed in \secref{section:requirements}.

\subsection{Machine Learning}
Machine learning is a technique in which the system saves data and uses it to create or enhance a model.
Some factors in the application are unknown and could be found using machine learning to help improve dead reckoning.
An example of these factors could be the variance gain from \secref{section:normal-distribution}, which is currently a fixed estimated value but could vary for different phones.
Machine learning could be taxing for the phone and cause new problems, such as additional technology dependencies, excessive storage usage, and system delay.