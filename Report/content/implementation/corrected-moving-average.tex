\section{CorrectedMovingAverage}
The \textit{CorrectedMovingAverage} class adjusts the acceleration, such that a deceleration of a step is the same size as the acceleration of the step.
This adjustment is done with the help of a stack which saves each step of an acceleration, such that it can be used for the corresponding deceleration.
As mentioned in \secref{section:dead-reckoning-approaches}, it is a hack, since it does not follow the defined rules for updating the marginal probability distributions.
However, it proved to be a good solution that takes the velocity gain into account.

To track the direction of a step, an enumerator was created which had the elements \textit{Unknown}, \textit{Left}, and \textit{Right}. 
When the step direction is unknown, a step has not been detected yet and the program checks each acceleration to see if it is outside the jitter zone.
Jitter noise occurs when the user unintentionally shakes the phone.
The jitter zone is an interval specifying where jitter noise can occur.
When the acceleration is higher than the jitter zone, a right step has begun, and if it is lower, it is a left step.
This categorisation is done with the knowledge of how an acceleration graph for a step is shaped, and is found in \secref{section:accelerometer}.
In addition, when a step has been detected, the velocity is reset to zero as the step has just begun.

If a step is in progress, every next acceleration, which is in the same direction, is loaded into a stack.
Once the deceleration has begun, the stack is popped for each consecutive deceleration.
