\section{Network}
The Network class is implemented to keep track of the current timeslice of the dynamic Bayesian network. 
The Network contains nine properties corresponding to the nodes for a given timeslice.
A timeslice can be seen in \figref{figure:dynamic-bayesian-network}.

\begin{lstlisting}[caption={Constructor for Network}, label=lst:network-constructor, float=h, style=sharpc]
public Network()
{
	Acceleration = new Acceleration(Vector3.Zero);
	Gyroscope = new Gyroscope(Vector3.Zero);
	ATan2 = new ATan2Angle(Acceleration);
	ComplementaryFilter = new ComplementaryFilter();
	SensorFusion = new SensorFusion(ComplementaryFilter, Acceleration);
	MovingAverage = new MovingAverage(0.0f, 0.0f);
	CorrectedMovingAverage = new CorrectedMovingAverage();
	Velocity = new Velocity(0.0f, 0.0f);
	Position = new Position(0.0f, Constants.POSITION_START);
}
\end{lstlisting}

When the Network is initialised, all the variables of the Network get assigned to their default value as seen in \lstref{lst:network-constructor}.
The reason for setting the variables to their default values, when initialising the Network, is to have a fixed-point to update the next timeslice.
After the Network has been initialised, a similar method can then be used to update the network, for a new timeslice, corresponding to a new accelerometer reading.

\begin{lstlisting}[caption={UpdateNetwork method in Network}, label=lst:network-update-network, float=h, style=sharpc]
public void UpdateNetwork(Vector3 newAcceleration, Vector3 newGyroscope)
{
	Acceleration = new Acceleration(newAcceleration);
	Gyroscope = new Gyroscope(newGyroscope);
	ATan2 = new ATan2Angle(Acceleration);
	ComplementaryFilter = new ComplementaryFilter(ATan2, Gyroscope, ComplementaryFilter);
	SensorFusion = new SensorFusion(ComplementaryFilter, Acceleration);
	MovingAverage = new MovingAverage(MovingAverage, SensorFusion);
	CorrectedMovingAverage = new CorrectedMovingAverage(MovingAverage);
	Velocity = new Velocity(Velocity, CorrectedMovingAverage);
	Position = new Position(Position, Velocity);
}
\end{lstlisting}

\lstref{lst:network-update-network} shows the method for updating the network to a new timeslice.
The \lstinline$UpdateNetwork$ method updates the probabilities in the order of the dynamic Bayesian network, from parents to children, where each probability distribution update is handled in the given classes that are instantiated.
The \textit{Acceleration} class, which is part of the Network, will be described hereafter.