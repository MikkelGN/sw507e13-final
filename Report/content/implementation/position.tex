\section{Position}
The \textit{Position} class is used to track the position of the phone, and is a subclass of the Variable class.
The position has three parents, which is the previous position, current velocity, and the ball position corresponding to the dynamic Bayesian network described in \secref{section:dynamic-bayesian-network}.

\begin{lstlisting}[caption={Update mean method containing constraints for position}, label=lst:PositionConstraint, float=h, style=sharpc]
protected override void UpdateMean()
{
    Mean = Constants.POSITION_MEAN_GAIN + Constants.F_POSITION[0, 0] * Parent.Mean + Constants.F_POSITION[0, 1] * Previous.Mean;

    if (Mean > Constants.POSITION_MAX || Mean < Constants.POSITION_MIN)
    {
        if (Mean > Constants.POSITION_MAX)
            Mean = Constants.POSITION_MAX;
        else
            Mean = Constants.POSITION_MIN;    
        
        Variance = 0.0f;
        Parent.Mean = 0.0f;
        Parent.Variance = 0.0f;
    }
}
\end{lstlisting}

In \lstref{lst:PositionConstraint}, the \lstinline$UpdateMean$ method has been implemented to take all parents into account for the Mean, as can be seen on line 3.
However, a constraint on the mean of the position has also been implemented, which is the limit of the area, corresponding to walking into the wall, see \figref{figure:fixed-movement-area} and can be seen in line 5.
The constraint resets the variance of the position in line 12, and resets the mean and variance of the velocity in line 13-14, and is implemented to handle uncertainties in the positioning measurement.