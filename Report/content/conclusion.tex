\chapter{Conclusion}
To conclude this project, each of the three questions have to be answered to confirm or invalidate the hypothesis.

The first question was:

\begin{center}
\textit{What are the requirements for a smartphone application that make people want to exercise?}
\end{center}
It was found that the requirements were primarily \textit{Exercise}, \textit{Dead Reckoning}, and \textit{Game}.
All of these requirements were implemented in the application, in the form of a game using dead reckoning to track the user's movement. 
The movement was used as a controller for the game, which made the user exercise.
It is believed that the game would make people want to exercise by playing the game, as using movement to play games have proven to be fun for other applications, such as the \textit{Nintendo Wii} and \textit{Xbox Kinect}.
Furthermore, the movement tracking as controller can easily be used in a wide range of other applications, such as various old arcade games.
As of such, there is a lot of options to use the movement tracking for a given user that may like one game and not another.

Whether the sidestepping movement is considered exercise is up for discussion. 
However, if exercise is defined as any form of body movement, it is considered exercise.

The second question was:

\begin{center}
\textit{How can a smartphone application motivate people to exercise?}
\end{center}

% Game elements are motivating - check
% Is an entertaining game motivating - (check)
% Exercise everywhere
% Special controller
% Ubiquitous??? 

A smartphone application has been developed that may motivate people to exercise.
Game elements, such as a high score, power-ups, and increasing difficulty, have been implemented, which were assumed to make the application fun and thus motivating for people.
Furthermore, these elements were implemented to give some replay value, such that people would continue to play the game.
Whether an entertaining game motivates people to exercise or not is debatable and subjective, see \secref{section:motivational-factors}.

In addition, the game can be played everywhere and does not require information from external hardware, such as Wi-Fi and GPS, as it works if the user has a smartphone with an accelerometer and gyroscope.

The fact that the application utilises the user movement as the controller can make for a fun and engaging experience, which in turn can be a motivational factor to continue playing the game. 

The third question was:

\begin{center}
\textit{Why would people use this application instead of other applications?}
\end{center}

As an intensive exercise application the user should not use this application over other applications, such as Endomondo and Zombies, Run!.
However, it is believed with the current state of the application that it serves as a fun and engaging experience that may serve as a  foundation for more intensive exercise.
Furthermore, the application serves as an alternative to other applications on the market.
The application developed focuses on the game as a way to exercise, whereas other applications make exercise the central part of the application.


Finally, the hypothesis was:

\begin{center}
\textit{A game that is fun and motivates the user to exercise can be developed for a smartphone using dead reckoning.}
\end{center}

A game has been developed for a smartphone that uses dead reckoning.
Furthermore, the dead reckoning works, but some adjustments are needed to make the tracking more accurate.
The question that remains is, if the game developed is fun and motivates the user to exercise.
It is presumed that the game is fun and motivating for some people, and would be a good basis for the application, but this assumption would have to be examined. 