\section{Existing Systems}\label{section:existing-systems}
To get inspiration for this project, three existing systems have been examined.
They are called \textit{Endomondo}, \textit{Zombies, Run!}, and \textit{Moves}.
The following examination of the applications acts as a source of inspiration throughout development of the project.

\subsection{Endomondo}
Endomondo is a fitness tracker combined with a social network solution, which runs on smartphones.
When using Endomondo, an activity is selected, such as running or bicycling. 
When an activity is started, the route is tracked.
The tracked information available is as follows \citep{misc:endomondo}:
\begin{itemize}
\item Overview of the route
\item Distance
\item Duration
\item Average speed
\item Maximum speed
\item Maximum/Minimum altitude
\item Total ascent/descent
\item Calories burned
\end{itemize}

All of this information can be shared on the Endomondo website and other social media.

\subsection{Zombies, Run!}
Zombies, Run! turns normal runs into small adventures in a post apocalyptic zombie world \citep{misc:zombiesrun}.
The adventure is narrated by different characters met in the storyline.
In the game the player is called \textit{runner 5} and is part of a group of runners, who run outside the safety zone of \textit{Abel Township} to collect resources for the town to ensure its survival.
When outside the city the runner can be spotted by zombies, this event is called \textit{Zombie Chase} which is an optional feature. 
If the runner is spotted by a zombie he has to run 20\% faster over a period of one minute to escape. 
When the user is not exercising, he can play a game where the resources are spent to improve Abel Township.
The user can choose to have the route tracked by GPS or the steps determined by use of the accelerometer.

\textit{Six to Start}, which is the company that created Zombies, Run!, provides a website with further information about the application \citep{misc:zombiesrun}.
When a user creates an account for the website, he can synchronise the data from his phone with the website. 
The website presents additional data collected from runs which are not visible in the application, a sample of the information present on the website is as follows:

\begin{itemize}
\item Overview of the route
\item Distance
\item Duration
\item Episodes completed
\item Average speed
\item Maximum/Minimum speed
\end{itemize}

Having the information available on the website is an optional feature that allows the user to track his progress.


%https://play.google.com/store/apps/details?id=com.sixtostart.zombiesrun&hl=da 
%https://itunes.apple.com/us/app/zombies-run!/id503519713?mt=8 
%https://www.zombiesrungame.com/ 
%This information section of the app it self

\subsection{Moves}
The application Moves \citep{misc:Moves} uses the GPS and accelerometer for registering and counting steps taken by the user.
According to their website, Moves will distinguish between walking, running and bicycling, as long as the user keeps the smartphone in a bag or a pocket.
The user can give Moves information about his weight, height, age, and gender. 
With these physical characteristics, Moves can calculate the amount of calories burned throughout the day.
Moves has a set of features that allows the user to view a summary of his tracked progress, as well as the route that he travels throughout the day.
However, it is not possible to export data from the Moves application itself to other devices, but it is possible to utilise a third party application for this purpose.

%The application was tested, in reality the tracking system was faulty

%https://moves-app.com/

\subsection{Summary}
The tracked information provided by the different applications provide some ideas of what could be recorded and shown. 
As an example it would be interesting to give an approximation of the calories burned.
The velocity information would be more interesting in an application where it is required to run, than in an application that focuses on the user strafing back and forth or performing push-ups.

Some features from Endomondo that might be interesting for this project is the ability to share results with friends, or to compare a result with previous results in order to get motivation and to track the users progress.

In Zombies, Run! there is a story which engages the user into running. 
It is entertaining and motivates the user to use the application more often.
Another good feature in Zombies, Run! is to view the speed of the user at any given time.

All of these features and ideas will be considered to be a part of the application for this project.
Additional approaches on how to motivate a user will be explained in \secref{section:motivation}.