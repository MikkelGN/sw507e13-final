\section{Movement Tracking} \label{section:movement-tracking}
In this section, commonly used positioning systems available for modern smartphones will be examined. 
The knowledge from the examination will be used to get a better understanding of how indoor and outdoor positioning works.

\subsection{Global Positioning System} \label{section:gps}
Global Positioning System (GPS) is a system developed by and for the US military, but has proven to be highly usable in civilian applications as well. 
GPS is mainly used for navigation and tracking, and is able to pinpoint a receiver's location to an accuracy of a few meters.
For the system to function accurately, the receiver must have signal from at least four orbiting satellites. 
However, when a user enters a building or a tunnel, he is unable to maintains a signal, and thereby the system is unable to function properly.
In recent years, research and studies have enabled GPS-receivers to function with only three satellites. 
It is accomplished by looking at previously collected data, using other available sensors and basically make an educated guess or estimate the receivers current location, which is called \textit{dead reckoning} \citep{Misc:WhatIsGPS}. 
If dead reckoning is used in conjunction with the GPS, the accuracy is increased. \citep{Article:JesseDeadReckoning, Article:PedestrianDR, PhdThesis:IndoorLocalization, Article:DRParameterCorrection}

\subsection{Dead Reckoning}\label{section:dead-reckoning}
Dead reckoning (DR) is a technique used to calculate the position of a device, which for this project is a smartphone.
DR uses the previous stored positions and calculates the next position based upon estimated speed, elapsed time, and the direction itself. 
DR is commonly used for the purpose of indoor positioning and tracking. 
DR combined with GPS is known to be more accurate than using GPS alone, this is done by utilising inertial measurement unit (IMU) \citep{Article:WirelessAssistedPDR, Article:PedestrianNavigationSystem}.
Some sensors that can be used for DR are the accelerometer, gyroscope, and compass as it is integrated in various smartphones. 
DR is also used for ships and aircraft \citep{Misc:DRBritannica}, but the focus will lie on smartphones and therefore the description of the DR techniques hereafter is applicable to smartphones.

DR can be implemented as a step length estimator, where the accelerometer serves as a pedometer, and the direction of the smartphone can be determined by use of a gyroscope and magnetometer.
However, a few problems exists when using DR.
The first problem is that sensor data can be flawed. 
For example the object could jump in place resulting in the accelerometer, gyroscope, or the compass data being misinterpreted. 
It can lead to false positions that are analysed for the next step.

A central problem with DR is the measurement uncertainty of the position, since various factors can disturb the calculated position. 
Some of the issues could be how the device is held, step size, and direction of movement.
These uncertainties can prove to be inaccurate, since if one step is calculated wrong the next step will include the error.
To minimise the measurement uncertainties, different constraints can be applied. 
These constraints can be holding the device in a specific way, restricting movement space, and limiting the movement pattern.
%If with those constraints the positioning proves to be fairly accurate, the ways that are allowed to be moved can be expanded. SKAL SKRIVES I LIMITATION IK HER 

To summarise, GPS gives an accurate position when receiving signal from four or more satellites. 
DR can be used in conjunction with GPS and maintain the position when the signal to satellites are temporarily lost. 
However, DR can also be used without GPS, but is prone to errors. 

\subsection{Wi-Fi Position}\label{subsection:wi-fi}
Wi-Fi positioning techniques works by having access to one or more Wi-Fi access points.
Depending on indoor or outdoor positioning there are different techniques in finding the position of the user \citep{Techreport:Robin}.

When using \textit{fingerprinting}, the received signal strength (RSS) from access points is recorded and a radio map is constructed from this data.

It means that a map of the environment is created with fixed points, containing RSS values of the access points, strategically scattered throughout the map. 
Each point has a value of RSS from a number of Wi-Fi access points, these RSS values are then used to calculate the location of a device using fingerprinting and the radio map.
There is no need for hardware modification to make fingerprinting work and it is accurate enough to place people in the correct rooms in a building.
However, creating the radio map can be a very time consuming process, due to the complex preparation of creating the radio map. 
The technique works both outside and inside, because the technique only needs the IDs of the relative RSS values of the access points in the radio map.

Angle of Arrival (AoA) is a method that uses the signals from access points and determines a location using geometry, based on the location of the access points. 
The device needs a special antenna, which needs to be able to calculate the angle in which the signal is received.
With two or more Wi-Fi access points, the position can be calculated by using the angles from which the Wi-Fi signal is transmitted.
\textit{Triangulation}, the use of trigonometry and geometry, can be used to further enhance the accuracy of AoA. 
The disadvantage of AoA is the need of special hardware antennas.  
The advantage is that it only needs two signals from access points to calculate a position.
 
Cell Identity (CI) is a technique where the positions of access points needs to be known. 
It is assumed that the strongest signal is from the closest access point.
Using this information, it is assumed that the device is close to the strongest signal and an approximation of the position is determined. 
It can give bad results when used indoor because of obstacles blocking the signal, and thereby making an access point further away have the strongest signal.
An advantage is that CI gives a fast initial approximation of the position.