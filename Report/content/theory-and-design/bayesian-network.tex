\section{Bayesian Network}\label{section:bayesian-network}
A Bayesian network is a directed acyclic graph, where each edge shows how a parent has an influence on its child.
The Bayesian network, see \figref{figure:simple-Bayesian-network}, shows how the different variables affect each other.
In the figure there are three different types of variables.
The first variable $A$, which is shaded, is an information variable. 
Information variables, can be either stochastic or deterministic, hold information that are obtained from input-sources.
The second variable $B$ is a deterministic variable, marked with two circles, the data in this variable type is based on parent values, and has no probability distribution.
Lastly, the variable $C$ with a single circle, is called a stochastic variable. 
Stochastic variables can be either continuous or discrete.
Continuous variables can hold an infinite number of values between two points, while discrete is a finite set.
As an example, a continuous variable holding a value between two points, 0 and 1, can have any number between these two points, which is an infinite set of real numbers.

\begin{figure}[H]
    \centering
    \includegraphics[scale=0.35]{media/Bayesian-network-abc}
    \caption{A simple Bayesian network}
    \label{figure:simple-Bayesian-network}
\end{figure}

When constructing a Bayesian network the following must be considered: 
\begin{itemize}
    \item What are the relevant variables? 
    \item What values types should be in the network?
    \item What is the relationship between the different variables?
\end{itemize}