This project is a semester project for embedded systems, where machine intelligence has been used to track movement by utilising dead reckoning techniques. 
The goal of the project is to develop a game application, for a smartphone, that can motivate people to indirectly do exercise, using only the integrated sensors of the smartphone.
While it proves harder to rely only on the integrated sensors, it gives the advantage that it can be used anywhere, as the approach does not require external information from GPS, Wi-Fi, etc. 

By trial and error, it was discovered that the data from the integrated sensors were inaccurate. 
This inaccuracy lead to imprecise estimation of the current position of the phone, which meant the game was unplayable. 
In order to solve this issue, various filters and probability assessments, at different stages of the application were implemented.
After the corrections were implemented, the application was more accurate, as the positioning no longer drifted. 
In conclusion, it has been demonstrated that it is possible to rely only on the data from the integrated sensors of a smartphone, to create a game application in which a user can perform exercise.